\documentclass{article}

\usepackage{amsmath}
\usepackage{amsthm}
\usepackage{amsfonts}
\usepackage{framed}

%% Simple AMSthm environments, numbered together.
\newtheorem{lem}{Lemma}
\newtheorem{thm}[lem]{Theorem}
\newtheorem{conj}[lem]{Conjecture}

\theoremstyle{definition}
\newtheorem{defn}[lem]{Definition}

\begin{document}

\subsubsection{Mathematical Flow Functions}
The available expressions analysis is a ``must'' analysis. Thus, we
define the lattice to be $(D, \sqsubseteq, \top, \perp, \sqcup,
\sqcap)$ = $(2^A, \supseteq, A, \emptyset, \cap, \cup)$ where

\[ A = \{x \rightarrow N~|~x \in \textup{Values} \wedge x \in \textup{Instructions}\} \].

Values and instructions are the LLVM types corrosponding to Value
pointers and instruction pointers, respectively. The idea is that
instructions are the set of expressions in LLVM.

The flow functions implemented in our framework are as follows.

\begin{center}
\begin{tabular}{c | c} % centered columns (4 columns) 
Function & Output \\ [0.5ex] % inserts table 
%heading 
\hline % inserts single horizontal line 
$F_{X:= Y \text{op} Z}(in)$ & $in \cup \{ X \rightarrow W : \exists \langle W, E\rangle \in in.  E = Y \text{op} Z\} \cup$ \\
& $\{ X \rightarrow Y \text{op} Z : \neg\exists \langle W, E\rangle \in in.  E = Y \text{op} Z\}$\\
$F_{if(*) \text{true-branch}}(in)$ & $in$ \\ 
$F_{if(*) \text{false-branch}}(in)$ & $in$ \\ 
$F_{X := \varphi(Y,Z)}(in_1, in_2)$ & $(in_1 \cap in_2) \cup \{X \rightarrow E : \langle Y, E \rangle \in in_1 \wedge \langle Z,E \rangle \in in_2 \}$\\
$F_{X := \text{cmp}(Z,Y)}(in)$ & $in$\\
 [1ex] % [1ex] adds vertical space 
\hline %inserts single line 
\end{tabular} 
\end{center}
\subsubsection{Implementation Considerations}
Most of the implementation of available expression analysis flow
functions was straightforward. There were, for example, no issues with
potentially infinite sets because each program is a fintie set of
expressions. The data structures used were simple
\texttt{std::maps}. 

The most involved thing we implemented was correctly deciding
$\varphi$ nodes that have arbitrary numbers of arguments, not just a
pair of arguments. This is not represented in the mathematical
flowfunction above (the notation was too cumbersome), but our
implementation handles $n$-ary $\varphi$ nodes by overlapping pairwise
comparisons of all operands. 

The most challenging aspect of this implementation would have been
comparing distinct instructions to see whether they are semantically
equal or not. Our implementation does the easiest thing possible,
which is calling the member function of the Instruction class
\texttt{isIdenticalToWhenDefined}. Unfortunately, this member function
only decides syntactic equivalence. So, for example, a commutative
operator like \texttt{add} will not compare equal if the operands are
in the wrong order.

If we had more time, we would have added some extra intelligence to
the equivalence test. For example, for commutative operators it would
be conceptually simple to swap operands into lexicographic order by
pointer and then compare.

\end{document}

