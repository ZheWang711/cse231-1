\documentclass{article}

\usepackage{amsmath}
\usepackage{amsthm}
\usepackage{amsfonts}
\usepackage{framed}

%% Simple AMSthm environments, numbered together.
\newtheorem{lem}{Lemma}
\newtheorem{thm}[lem]{Theorem}
\newtheorem{conj}[lem]{Conjecture}

\theoremstyle{definition}
\newtheorem{defn}[lem]{Definition}

\title{Project 2 Report}
\begin{document}
\maketitle

\section{Overview}
\begin{framed}
  Discuss the high level goals of your work, along with any
  interesting/key findings.
\end{framed}


Difficult, because there were no interesting findings. Consider:
``huh, LLVM has an API for some of this stuff I guess,'' or ``Wow C++
is painful.'' I (marco) will be able to make up some stuff for this
once I'm done with CSE (which should soon).

\section{Interface Design}
\begin{framed}
  Describe interface. Discuss what alternative designs you may have
  also considered, and explain the tradeoffs that ultimately led to
  your choice.
\end{framed}

Unsure how honest to be here. Tradeoffs ended up being, how many C++
features can we avoid and still have a functioning C++ program? Other
than this issue, describing our interface (flowfunction object,
latticepoint object, and a blob of code implementing the worklist
algorithm that uses runtime polymorphism to use these objects) is
pretty straightforward. I (marco) can write this.

\section{Analyses}

\subsection{Constant Propagation}
\subsubsection{Mathematical Flow Functions}
\begin{framed}
  define your lattice and flow functions in mathematical notation
  (i.e., in the style used in the lecture notes and on the midterm).
\end{framed}


\subsubsection{Implementation Considerations}
\begin{framed}
  discuss how you actually went about implementing the lattice and
  flow functions. For instance, some interesting questions are: what
  data structure(s) did you use, and how do you represent potentially
  infinite sets? How are input facts passed to your flow functions,
  and how do output facts get propagated?
\end{framed}
\subsection{Available Expressions}
\subsubsection{Mathematical Flow Functions}
\begin{framed}
  define your lattice and flow functions in mathematical notation
  (i.e., in the style used in the lecture notes and on the midterm).
\end{framed}
\subsubsection{Implementation Considerations}
\begin{framed}
  discuss how you actually went about implementing the lattice and
  flow functions. For instance, some interesting questions are: what
  data structure(s) did you use, and how do you represent potentially
  infinite sets? How are input facts passed to your flow functions,
  and how do output facts get propagated?
\end{framed}
\subsection{Range Analysis}
\subsubsection{Mathematical Flow Functions}
\begin{framed}
  define your lattice and flow functions in mathematical notation
  (i.e., in the style used in the lecture notes and on the midterm).
\end{framed}
\subsubsection{Implementation Considerations}
\begin{framed}
  discuss how you actually went about implementing the lattice and
  flow functions. For instance, some interesting questions are: what
  data structure(s) did you use, and how do you represent potentially
  infinite sets? How are input facts passed to your flow functions,
  and how do output facts get propagated?
\end{framed}
\subsection{Intra-Procedural Pointer Analysis}
\subsubsection{Mathematical Flow Functions}
\begin{framed}
  define your lattice and flow functions in mathematical notation
  (i.e., in the style used in the lecture notes and on the midterm).
\end{framed}
\subsubsection{Implementation Considerations}
\begin{framed}
  discuss how you actually went about implementing the lattice and
  flow functions. For instance, some interesting questions are: what
  data structure(s) did you use, and how do you represent potentially
  infinite sets? How are input facts passed to your flow functions,
  and how do output facts get propagated?
\end{framed}
\section{Testing}
\begin{framed}
  Make sure to explain assumptions you make about the code you
  analyze. For instance, for pointer analysis, you may have made some
  assumptions about the aliasing information known about input
  parameters. Explain those assumptions and why they are reasonable.

  Part of this project is to come up with a useful set of benchmarks
  on which to test and improve your analysis. Discuss why you chose
  those benchmarks, and what makes them interesting. If your
  implementation fails on some benchmarks (there's no shame in it!),
  then explain why and how the analysis might be improved.
\end{framed}
\subsection{Benchmarks/Assumptions in Common}
Because we have a common pool of benchmarks, we can list them here
along with common assumptions on code. More specialized discussion of
per-analysis benchmark goes below. Here, we can also introduce the
Straight Line Program/Branching Program distinction.

\subsection{Constant Propagation}
\subsection{Available Expressions}
\subsection{Range Analysis}
\subsection{Intra-Procedural Pointer Analysis}



\section{Conclusion/Challenges}
\begin{framed}
  As this project is significantly more exploratory than the first, we
  want you to tell us what you found particularly
  interesting/challenging/frustrating. What extensions to the project
  did you attempt? (for instance, did you try combining the results of
  analyses, or did you try your hand at interprocedural
  analysis?). What aspects of LLVM made it easy/hard to implement your
  design? If you could redo your project with what you know now, what
  changes would you make?
\end{framed}
Also unsure how honest to be here. Most of our challenges had to do
with counter-intuitive LLVM design and poor documentation, as well as
the unpredictability of C++ features.

\end{document}

