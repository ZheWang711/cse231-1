\documentclass{article}

\usepackage{amsmath}
\usepackage{amsthm}
\usepackage{amsfonts}
\usepackage{framed}

%% Simple AMSthm environments, numbered together.
\newtheorem{lem}{Lemma}
\newtheorem{thm}[lem]{Theorem}
\newtheorem{conj}[lem]{Conjecture}

\theoremstyle{definition}
\newtheorem{defn}[lem]{Definition}

\begin{document}

\subsubsection{Mathematical Flow Functions}

\begin{framed}
  define your lattice and flow functions in mathematical notation
  (i.e., in the style used in the lecture notes and on the midterm).
\end{framed}

Unlike the previous two analyses, range analysis is a `may' analysis. Thus the most conservative range for a variable is the full set and the most optimistic is the empty set. With these considerations in mind, let us define our lattice and flow functions. 

One may be tempted to write the domain as

\[ 2^{ \{ x \rightarrow [a , b] \; | \; x \in \text{Vars} \, \wedge \, -\infty \leq a \leq b \leq \infty \} }.  \]

While this has the correct intuition, it is quite imprecise, as we do not want there to be the possibility of a variable having more than one range. Given a variable $x \in$ Vars, let us define $D_x$ as the domain of $x$, i.e. 

\[ D_x =  2^{ \{ x \rightarrow [a , b] \; | \;  -\infty \leq a \leq b \leq \infty \} }. \]

Then our final domain is the Cartesian product of $D_x$'s, which we write as

\[ D = \prod_{x \in \text{Vars}} D_x. \]

Top will correspond to the product of full sets, that is if $x_1, x_2, \ldots$ is an enumeration of Vars, then

\[  \top = (x_1 \rightarrow [-\infty, \infty], x_2 \rightarrow [-\infty, \infty], \ldots). \]

Similarly, we have

\[ \bot =  (x_1 \rightarrow \emptyset , x_2 \rightarrow \emptyset, \ldots). \]

Given two elements of our domain, $A, B \in D$, we can write $A = (x_1 \rightarrow [a_1, b_1] , x_2 \rightarrow [a_2, b_2], \ldots)$ and $B = (x_1 \rightarrow [a'_1, b'_1] , x_2 \rightarrow [a'_2, b'_2], \ldots)$. Then $A \sqsubseteq B$ if and only if $[a_i, b_]) \subseteq [a'_i, b'_i]$ for all $i$. Additionally, if for all $i$, we let $\overline{a_i} = \max \{ a_i, a'_i \}$ and $\underline{a_i} = \min \{ a_i, a'_i \}$ (similarly for $\overline{b_i}$ and $\underline{b_i})$, then we can write

\[ A \sqcup B = (x_1 \rightarrow [\underline{a_i}, \overline{b_1} ], x_2 \rightarrow [\underline{a_2}, \overline{b_2} ] )\]

and

\[ A \sqcap B = (x_1 \rightarrow [\overline{a_i}, \underline{b_1} ], x_2 \rightarrow [\overline{a_2}, \underline{b_2} ] )\]

where we take the convention that $[a, b] = \emptyset$ if and only if $b < a$.



\subsubsection{Implementation Considerations}
\begin{framed}
  discuss how you actually went about implementing the lattice and
  flow functions. For instance, some interesting questions are: what
  data structure(s) did you use, and how do you represent potentially
  infinite sets? How are input facts passed to your flow functions,
  and how do output facts get propagated?
\end{framed}

\end{document}

