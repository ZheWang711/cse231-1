\documentclass{article}

\usepackage{amsmath}
\usepackage{amsthm}
\usepackage{amsfonts}
\usepackage{framed}

%% Simple AMSthm environments, numbered together.
\newtheorem{lem}{Lemma}
\newtheorem{thm}[lem]{Theorem}
\newtheorem{conj}[lem]{Conjecture}

\newcommand{\Z}{\mathbb{Z}}


\theoremstyle{definition}
\newtheorem{defn}[lem]{Definition}

\begin{document}

\subsubsection{Mathematical Flow Functions}

Unlike the previous two analyses, range analysis is a `may' analysis. Thus the most conservative range for a variable is the full set and the most optimistic is the empty set. With these considerations in mind, let us define our lattice and flow functions. The domain of the analysis is the set of maps from variables to the set of ranges specified by the extended integers (i.e. allowing $+\infty$ and $-\infty$). Thus,

\[ D = \text{Vars} \rightarrow \Z^{\infty} \times \Z^{\infty}. \]

For $A \in D$ and $x \in $ Vars, we will interchangeably think of $A(x) = [a, b]$ as a range (i.e. a set of numbers) and a tuple (a pair of numbers). With this reasoning, let top will correspond to the constant map of full sets, i.e. for any variable $x \in $ Vars,

\[  \top(x) =  [-\infty, \infty]. \]

Similarly, bottom corresponds to the constant map of empty sets, i.e. for any variable $x \in $ Vars,

\[  \bot(x) =  [\infty, -\infty]. \]

Here we take the convention that $[a, b] = \emptyset$ if and only if $b < a$. Given two elements of our domain, $A, B \in D$, we write $A \sqsubseteq B$ if and only if $A(x) \subseteq B(x)$ for all $x \in$ Vars. We can also say for any $x \in $ Vars,

\[ A \sqcup B(x) = [\min \{ \underline{A(x)}, \underline{B(x)} \} , \max \{  \overline{A(x)}, \overline{B(x)} \} ]\]

\noindent and

\[ A \sqcap B(x) = [\max \{  \underline{A(x)} , \underline{B(x)} \} , \min \{ \overline{A(x)} , \overline{B(x)} \} ]\]

\noindent where we take the convention that $\overline{[a, b]} = b$ and $\underline{[a, b]} = a$. Now that the lattice has been defined, let's turn our attention to the various flow functions. To make the analysis more concrete, given an element $A \in D$ and variable $x$, let $A[x \rightarrow [a', b']]$ denote the exact same element of $D$ with the exception that $A(x) = [a', b']$. Now we enumerate the various flow functions.

\[ F_{X:= C}(in) = in[X \rightarrow [C, C]] \]

\[ F_{X:= Y \text{op} Z}(in) = in[X \rightarrow [\min L, \max L]]\text{ where } L = \{ a \text{ op } b \, | \, a \in in(Y) \wedge b \in in(Z)  \}\]

\[ F_{if( X \leq C) \text{ true-branch}}(in) = in[X \rightarrow [-\infty, C] \cap in(X)  ] \]

\[ F_{if( X \leq C) \text{ false-branch}}(in) = in[X \rightarrow [C + 1, \infty] \cap in(X)  ]\]

\[ F_{if( X == C) \text{ true-branch}}(in) = in[X \rightarrow [C, C] \cap in(X)  ]\]

\[ F_{if( X == C) \text{ false-branch}}(in) = in \]

\[ F_{if( X < C) \text{ true-branch}}(in) =in[X \rightarrow [-\infty, C -1] \cap in(X)  ] \]

\[ F_{if( X < C) \text{ false-branch}}(in) =  in[X \rightarrow [C, \infty] \cap in(X)  ]\]

\[ F_{if( X \geq C) \text{ true-branch}}(in) =  F_{if( X < C) \text{ false-branch}}(in) \]

\[ F_{if( X \geq C) \text{ false-branch}}(in) =  F_{if( X < C) \text{ true-branch}}(in)\]

\[ F_{if( X > C) \text{ true-branch}}(in) =   F_{if( X \leq C) \text{ false-branch}}(in)\]

\[ F_{if( X > C) \text{ false-branch}}(in) =   F_{if( X \leq C) \text{ true-branch}}(in)\]


\subsubsection{Implementation Considerations}
\begin{framed}
  discuss how you actually went about implementing the lattice and
  flow functions. For instance, some interesting questions are: what
  data structure(s) did you use, and how do you represent potentially
  infinite sets? How are input facts passed to your flow functions,
  and how do output facts get propagated?
\end{framed}

LLVM has a nice data structure for handling ranges called ConstantRange. This is a one-side inclusive interval $[a, b)$ that is meant to be taken as intersection with the integers. ConstantRange even has some convenience functions that make computing some of the above flow functions more bearable. 

As with the above analyses, we explicitly specify whether or not a lattice point is top or bottom with boolean variables, and we use algebraic identities when working with them. 

\end{document}

