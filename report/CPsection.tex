\documentclass{article}

\usepackage{amsmath}
\usepackage{amsthm}
\usepackage{amsfonts}
\usepackage{framed}

%% Simple AMSthm environments, numbered together.
\newtheorem{lem}{Lemma}
\newtheorem{thm}[lem]{Theorem}
\newtheorem{conj}[lem]{Conjecture}

\theoremstyle{definition}
\newtheorem{defn}[lem]{Definition}

\begin{document}

\subsubsection{Mathematical Flow Functions}

\begin{framed}
  define your lattice and flow functions in mathematical notation
  (i.e., in the style used in the lecture notes and on the midterm).
  
\end{framed}
The constant propagation analysis is a `must' analysis. Thus, we define the lattice to be $(D, \sqsubseteq, \top, \perp, \sqcup, \sqcap)$ = $(2^A, \supseteq, A, \emptyset, \cap, \cup)$ where

\[ A = \{x \rightarrow N~|~x \in \textup{Vars} \wedge x \in \mathbb{Z}\} \].

The flow functions implemented in our framework are as follows:

\[ F_{X:= Y \text{op} Z}(in) = in - \{X \rightarrow \ast\} \cup \{ X \rightarrow N~|~(Y \rightarrow N_1) \in in~\wedge ~(Z \rightarrow N_2) \in in~\wedge ~N = N_1~op~N_2 \} \]

\[ F_{if(X == C) \text{true-branch}}(in) = in - \{X \rightarrow \ast\} \cup \{ X \rightarrow C \} \]

\[ F_{if(X != C) \text{false-branch}}(in) = in - \{X \rightarrow \ast\} \cup \{ X \rightarrow C \} \]

At a join (i.e. PHI node), we take the intersection between the two incoming lattice points.

\[ A \sqcup B = A \cap B\]

\subsubsection{Implementation Considerations}
\begin{framed}
  discuss how you actually went about implementing the lattice and
  flow functions. For instance, some interesting questions are: what
  data structure(s) did you use, and how do you represent potentially
  infinite sets? How are input facts passed to your flow functions,
  and how do output facts get propagated?
\end{framed}

\end{document}

