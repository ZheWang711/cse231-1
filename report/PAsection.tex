\documentclass{article}

\usepackage{amsmath}
\usepackage{amsthm}
\usepackage{amsfonts}
\usepackage{framed}

%% Simple AMSthm environments, numbered together.
\newtheorem{lem}{Lemma}
\newtheorem{thm}[lem]{Theorem}
\newtheorem{conj}[lem]{Conjecture}

\theoremstyle{definition}
\newtheorem{defn}[lem]{Definition}

\begin{document}

\subsubsection{Mathematical Flow Functions}
Like range analysis, pointer analysis is a `may' analysis. The domain of the analysis can be written as

\[ D = 2^{\{ x \rightarrow y \; |  \; x \in \text{Vars} \, \wedge \, y \in \text{Vars} \}}. \]

Top and bottom are the full set and the empty set, respectively. Thus, $top =\{ x \rightarrow y \; |  \; x \in \text{Vars} \, \wedge \, y \in \text{Vars} \}  $ and $\bot = \emptyset.$ Finally, we have $\sqsubseteq \, = \, \subseteq$, $\sqcup = \cup$, and $\sqcap = \cap$. The flow functions for pointer analysis are displayed in the following table.

\begin{center}
\begin{tabular}{c | c} % centered columns (4 columns) 
Function & Output \\ [0.5ex] % inserts table 
%heading 
\hline % inserts single horizontal line 
$F_{X = C}(in)$ & $in \setminus \{ X \rightarrow * \}$ \\ % inserting body of the table 
$F_{X = Y + Z}(in)$ & $in \setminus \{ X \rightarrow * \}$ \\ 
$F_{X = Y}(in)$ & $in \setminus \{ X \rightarrow * \} \cup \{ X \rightarrow Z \, | \, Y \rightarrow Z \in in \}$ \\ 
$F_{X =  \& Y}(in)$ & $in \setminus \{ X \rightarrow * \} \cup \{ X \rightarrow Y \}$\\ 
$F_{X = *Y}(in)$ & $in \setminus \{ X \rightarrow * \} \cup \{ X \rightarrow B \; | \; \exists A \in \text{Vars s.t. }  Y \rightarrow A \in in \, \wedge \, A \rightarrow B \in in \}$\\
$F_{*X = Y}(in)$ & $in \setminus \{ X \rightarrow * \} \cup \{ A \rightarrow B \; | \; \exists A, B \in \text{Vars s.t. } X \rightarrow A \in in \, \wedge \, Y \rightarrow B \in in\}$\\
$F_{merge}(in_1, in_2)$ & $in_1 \sqcup in_2$\\ [1ex] % [1ex] adds vertical space 
\hline %inserts single line 
\end{tabular} 
\end{center}

\subsubsection{Implementation Considerations}

Like the rest of the analyses, we handle top and bottom with boolean algebras. For the rest of the cases, a lattice point consists of a set of Value pointers that could potentially point to any variable and of a map from Value pointers to sets of Value pointers which are meant to be taken as the variables that this Value could point to. 

To actually implement the above flow functions, we had to figure out how to map from LLVM operations (such as load and store) to C-style semantics that we covered in class. When in doubt, we erred on the side of caution. Thus, we interpreted statements such as $store \; y \; x$ as representing both $*x = y$ and $x = \& y$, and implemented the flow functions accordingly. LLVM does have a type system, which we used in order to maintain a stronger sense of precision, but our intuition told us to sacrifice precision for correctness.

\end{document}

